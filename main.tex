\documentclass{article}
\usepackage[utf8]{inputenc}
\usepackage[spanish]{babel}
\usepackage{listings}
\usepackage{graphicx}
\graphicspath{ {images/} }
\usepackage{cite}

\begin{document}

\begin{titlepage}
    \begin{center}
        \vspace*{1cm}
            
        \Huge
        \textbf{INFORMATICA II}
            
        \vspace{0.5cm}
        \LARGE
        Calistenia 
            
        \vspace{1.5cm}
            
        \textbf{JUAN SEBASTIAN GARAVITO GALLO}
            
        \vfill
            
        \vspace{0.8cm}
            
        \Large
        Despartamento de Ingeniería Electrónica y Telecomunicaciones\\
        Universidad de Antioquia\\
        Medellín\\
        Marzo de 2021
            
    \end{center}
\end{titlepage}

\newpage
\section{Introduccion}\label{intro}
Siguiendo una serie de instrucciones trataremos de llevar un objeto de una posicion A a una posicion B.

\section{Contenido} \label{contenido}
Los elementos que necesitamos para realizar la actividad son, dos tarjetas de similar peso y una hoja de papel.
Reglas: Se deberan de llevar las tarjetas de la posicion inicial (Una encima de la otra) hasta la posicion final (una "casita" entre ellas) todo esto con una sola mano y encima de la hoja de papel. No pueden haber objetos externos que ayuden a sostener el equilibrio entre las tarjetas.


\begin{intemize}
 \item Pasos a seguir:
 \item 1)Colocar la hoja en una superficie plana.
 \item 2)Colocar una tarjeta encima de la otra, luego ubicarlas en el centro de la hoja.
 \item 3)Ubicar la palma de la mano encima de las tarjetas de manera que los dedos queden libres.
 \item 4)Arrugar la hoja con las yemas de los dedos, teniendo siempre sujetas las tarjetas en la palma de la mano ayudandonos de la raiz de los dedos para sostenerlas.
 \item 5)Estirar nuevamente la hoja sin soltar en ningun momento las tarjetas.
 \item 6)Cuando la hoja este estriada con las tarjetas en la mano ubicarse nuevamente en el centro de la hoja.
 \item 7)Para las tajertas de forma que una de las caras quede mirando hacia ti.
 \item 8)Ubicar el dedo indice en la parte superior de las tarjetas generandoles una leve presion.
 \item 9)Con el resto de los dedos de tu mano tomar las partes inferiores de una de las tarjetas.
 \item 10)Comenzar lentamente a separar la tarjeta que tomaste en el punto anterior (Llevandola hacia ti) todo esto sin dejar de realizar la presion con el dedo indice.
 \item 11)Formar "la casita" o el triangulo entre las tarjetas en perfecto equlibrio (que se sotengan entre si).
 \item 12)En caso de no encontrar el equilibrio regresar al paso numero 7.
\end{intemize}



\end{document}
